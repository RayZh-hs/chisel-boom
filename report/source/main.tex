%%%%%%%%%%%%%%%%%%%%%%%%%%%%%%%%%%%%%%%%%
% Wenneker Assignment
% LaTeX Template
% Version 2.0 (12/1/2019)
%
% This template originates from:
% http://www.LaTeXTemplates.com
%
% Authors:
% Vel (vel@LaTeXTemplates.com)
% Frits Wenneker
%
% License:
% CC BY-NC-SA 3.0 (http://creativecommons.org/licenses/by-nc-sa/3.0/)
% 
%%%%%%%%%%%%%%%%%%%%%%%%%%%%%%%%%%%%%%%%%

%----------------------------------------------------------------------------------------
%	PACKAGES AND OTHER DOCUMENT CONFIGURATIONS
%----------------------------------------------------------------------------------------

\documentclass[12pt]{scrartcl} % Font size

\usepackage{xcolor}
\usepackage{hyperref}

%%%%%%%%%%%%%%%%%%%%%%%%%%%%%%%%%%%%%%%%%
% Wenneker Assignment
% Structure Specification File
% Version 2.0 (12/1/2019)
%
% This template originates from:
% http://www.LaTeXTemplates.com
%
% Authors:
% Vel (vel@LaTeXTemplates.com)
% Frits Wenneker
%
% License:
% CC BY-NC-SA 3.0 (http://creativecommons.org/licenses/by-nc-sa/3.0/)
% 
%%%%%%%%%%%%%%%%%%%%%%%%%%%%%%%%%%%%%%%%%

%----------------------------------------------------------------------------------------
%	PACKAGES AND OTHER DOCUMENT CONFIGURATIONS
%----------------------------------------------------------------------------------------

\usepackage{amsmath, amsfonts, amsthm, longtable, multirow} % Math packages

\usepackage[english]{babel} % English language hyphenation

\usepackage{graphicx} % Required for inserting images
\graphicspath{{Figures/}{./}} % Specifies where to look for included images (trailing slash required)

\usepackage{booktabs} % Required for better horizontal rules in tables

\numberwithin{equation}{section} % Number equations within sections (i.e. 1.1, 1.2, 2.1, 2.2 instead of 1, 2, 3, 4)
\numberwithin{figure}{section} % Number figures within sections (i.e. 1.1, 1.2, 2.1, 2.2 instead of 1, 2, 3, 4)
\numberwithin{table}{section} % Number tables within sections (i.e. 1.1, 1.2, 2.1, 2.2 instead of 1, 2, 3, 4)

\setlength\parindent{0pt} % Removes all indentation from paragraphs

\usepackage{enumitem} % Required for list customization
\setlist{noitemsep} % No spacing between list items

%----------------------------------------------------------------------------------------
%	DOCUMENT MARGINS
%----------------------------------------------------------------------------------------

\usepackage{geometry} % Required for adjusting page dimensions and margins

\geometry{
	paper=a4paper, % Paper size, change to letterpaper for US letter size
	top=2.5cm, % Top margin
	bottom=3cm, % Bottom margin
	left=3cm, % Left margin
	right=3cm, % Right margin
	headheight=0.75cm, % Header height
	footskip=1.5cm, % Space from the bottom margin to the baseline of the footer
	headsep=0.75cm, % Space from the top margin to the baseline of the header
	%showframe, % Uncomment to show how the type block is set on the page
}

%----------------------------------------------------------------------------------------
%	FONTS
%----------------------------------------------------------------------------------------

\usepackage[utf8]{inputenc} % Required for inputting international characters
\usepackage[T1]{fontenc} % Use 8-bit encoding

\usepackage{fourier-otf} % Use the Adobe Utopia font for the document

\usepackage{caption}
\DeclareCaptionFont{lmss}{\fontfamily{lmss}\selectfont}

\captionsetup{
  	font={small,lmss},
  	labelfont={small,lmss}
}

%----------------------------------------------------------------------------------------
%	SECTION TITLES (KOMA-Script)
%----------------------------------------------------------------------------------------

% Section titles: centered, small caps
\setkomafont{section}{
  	\LARGE\centering\normalfont\scshape
}

% Subsection titles: normal font
% Subsection titles: bold
\setkomafont{subsection}{
  	\Large\normalfont\bfseries
}

% Subsubsection titles: italic
\setkomafont{subsubsection}{
    \normalfont\itshape
}

% Paragraph titles: small caps
\setkomafont{paragraph}{
    \normalfont\scshape
}

% Spacing before section titles (replaces \vspace hack)
\RedeclareSectionCommand[
    afterskip=2.5ex
]{section}
\RedeclareSectionCommand[
    afterskip=1ex
]{subsection}
\RedeclareSectionCommand[
  beforeskip=1.5ex,
  afterskip=0.2ex
]{subsubsection}

\setparsizes{0pt}{0.75\baselineskip}{\parfillskip}

%----------------------------------------------------------------------------------------
%	HEADERS AND FOOTERS
%----------------------------------------------------------------------------------------

\usepackage{scrlayer-scrpage} % Required for customising headers and footers

\ohead*{} % Right header
\ihead*{} % Left header
\chead*{} % Centre header

\ofoot*{} % Right footer
\ifoot*{} % Left footer
\cfoot*{\pagemark} % Centre footer
 % Include the file specifying the document structure and custom commands

%----------------------------------------------------------------------------------------
%	TITLE SECTION
%----------------------------------------------------------------------------------------

\title{	
	\normalfont\normalsize
	\textsc{Shanghai Jiao Tong University}\\
	\vspace{25pt}
	\rule{\linewidth}{0.5pt}\\ % Thin top horizontal rule
	\vspace{20pt}
	{\huge Boom Lite CPU Report}\\ % The assignment title
	\vspace{12pt}
	\rule{\linewidth}{2pt}\\ % Thick bottom horizontal rule
	\vspace{12pt}
}

\author{Ruihang Zhang \quad Yuetian Wang}

\date{\normalsize\today}

\begin{document}

\maketitle

\section{Overview}

In the past month we have implemented, as the course project of Computer Architecture, a fully out-of-order RISC-V CPU named Boom Lite\footnote{The source code is available at \url{https://github.com/RayZh-hs/chisel-boom/}}.

The CPU's architecture is inspired by the Berkeley Out-of-Order Machine (BOOM)\cite{Celio:EECS-2015-167}, though we made considerable simplifications and modifications to the original design to fit our time constraints. The final result is a CPU with more than 10 stages that accomplishes:

\begin{enumerate}
	\item Full support for the RV32I instruction set, including \texttt{JAL}, \texttt{JALR}, and bit-masked memory access.
	\item Full support for the M extension, using Wallace tree multipliers and restoring division algorithms.
	\item Unified memory system based on DRAM, along with ICache and DCache support.
	\item Multi-hierarchy branch predictor, combining a two-bit saturating BTB and a RAS.
	\item RAT-based OoO implementation to cut down on data movements.
	\item Partial OoO memory access as well as MMIO IO support.
\end{enumerate}

The CPU is written in Chisel, and has been synthesized to Verilog using Chisel's built-in FIRRTL compiler. The CPU achieved a clock frequency of over 517MHz on ASAP7 Demo, occupying under 15,000 um$^2$ of area.

\section{Design Details}


\section{Architecture}

\subsection{Frontend Design}

\begin{figure*}[h!]
	\centering
	\includegraphics[width=\textwidth]{figures/frontend.png}
	\caption{Boom Lite Frontend Diagram}
	\label{fig:frontend}
\end{figure*}

The frontend of Boom Lite consists of 5 stages, as shown in Figure \ref{fig:frontend}. The Fetch stage spans two cycles, the first of which fetches the instruction from ICache as well as consult the low-latency BTB for branch prediction. The second Fetch Stage gathers the fetched instruction and BTB feedback. If BTB predicts a taken branch, the PC is updated and the next-cycle fetch result will be killed.

The instruction is then passed on to RAS (Return Address Stack) Adaptor and the Decoder simultaneously. The Decoder simply performs combinational logic to retrieve fields from the instruction. The RAS Adaptor picks out \texttt{JAL} and \texttt{JALR} instructions, and takes action on the RAS.
According to common ABI patterns, \texttt{CALL} and \texttt{RET} commands are speculated and the RAS modified thus. The component produces surprisingly sound prediction results, as was discovered in the profiling phase. If the predicted destination conflicts with BTB predictions, the former is disregarded and the frontend before the decoder flushed. The results from the two components are merged in the PDP (Pre-Dispatch Plexer), and the PC, Decoded Bundle, Predicted Target and RAS Stack Pointer are passed on to Rename \& Dispatch phases.

The Rename phase consults the Free List for currently available physical registers and updates the Map Table (i.e. RAT, the Register Alias Table) accordingly, producing physical register indices for source and destination registers as well as documenting the stale pdst. The renamer also tracks the ROB ID for each instruction (which is deterministic since ROB is in-order enqueued).

The result is piped into a selective router known as the Dispatch Router, routing instructions to various Issue Buffers according to their types. We will cover the four issue buffers in the Backend Design section.

It is worth noting that the PC is \textbf{not} stored in the ROB. In fact, unless routed to the Branch Unit, the PC is discarded after the Dispatch phase. This design choice makes our ROB and normal issue buffers extremely small and efficient, having no need to pass around 32-bit information around.

\pagebreak
\bibliographystyle{plain}
\bibliography{references}

\end{document}
